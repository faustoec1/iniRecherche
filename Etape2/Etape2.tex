%----------------------------------------------------
%----------------   Trame Semaine 1  ----------------
%----------------------------------------------------

\documentclass[12pt]{article}
\usepackage[latin1]{inputenc}   %--- package pour les accents
\usepackage[T1]{fontenc}
\usepackage[french]{babel}      %--- package pour la langue
\usepackage{multicol}           %--- package pour plusieurs colonnes
\usepackage{graphicx}           %--- package pour inclure des figures



%--------------
\begin{document}

  \title{Titre du projet - Etat de l'art}
  \author{Auteurs}
  \maketitle


%----------------------------
\tableofcontents

%----------------------------
\section{Introduction}

D�crire la probl�matique de votre projet. 
Introduire la cat�gorisation de vos articles.


%----------------------------
\section{Etat de l'art}

R�diger l'�tat de l'art de votre recherche bibliographique en vous appuyant sur la cat�gorisation. 

Les r�f�rences bibliographiques sont � mettre dans un fichier {\tt bib} au format {\tt bibtex}. 
Pour les citer, utiliser {\tt $\backslash$cite}. 
La bibliography se g�n�re � la fin du document avec les commandes {\tt bibliographystyle} et {\tt bibliography}. 

Vous trouverez dans le fichier {\tt example.bib} des exemples de format pour un livre 
("Pour plus d'informations sur le r�f�rencement, consulter \cite{latexcompanion}"), 
un article de revue \cite{einstein}, un article de conf�rence\cite{viaud}, 
une th�se\cite{theseguillas}, un site web\cite{hal}.


\subsection{Cat�gorisation 1}

\subsubsection{Sous-cat�gorisation}

\subsection{Cat�gorisation 2}

\subsection{Cat�gorisation 3}



Illustrer avec un tableau comparatif


\begin{center}
\begin{tabular}{|c|c|c|p{2 cm}|}\hline
& type de m�thode & complexit� &  \\\hline
article \cite{latexcompanion} & 1 & $O(n)$ & r�sultats int�ressants \\\hline
article \cite{latexcompanion} & 2 & $O(nm)$ & m�thode pas claire \\\hline
\end{tabular} 
\end{center}

La figure \ref{fig} repr�sente des exemples de tableaux comparatifs extraits de l'�tat de l'art d'une th�se \cite{theseguillas}.
Les figures sont des �l�ments flottants que latex peut positionner sur n'importe quelle page. 
Pour le r�f�rencement des figures, utiliser {\tt label} et {\tt ref}. 


\begin{figure}
\includegraphics[width=0.40\textwidth]{ExempleTableau.png} \hspace*{2 cm}
\includegraphics[width=0.40\textwidth]{ExempleTableau.png} \hspace*{2 cm}
\caption{Exemples de tableaux comparatifs extraits de \cite{theseguillas}\label{fig}}
\end{figure}


%----------------------------
\section{Conclusion}




\bibliographystyle{alpha}
%\bibliographystyle{apalike}
%\bibliographystyle{plain}
\bibliography{example}

\end{document}
